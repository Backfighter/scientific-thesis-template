%Symbole
%--------
%\usepackage[geometry]{ifsym} % \BigSquare
%\usepackage{mathabx}
%\usepackage{stmaryrd} %fuer \ovee, \owedge, \otimes
%\usepackage{marvosym} %fuer \Writinghand %patched to not redefine \Rightarrow
%\usepackage{mathrsfs} %mittels \mathscr{} schoenen geschwungenen Buchstaben erzeugen
%\usepackage{calrsfs} %\mathcal{} ein bisserl dickeren buchstaben erzeugen - sieht net so gut aus.
                      %durch mathpazo ist das schon definiert
\usepackage{amssymb}

%name-clashes von marvosym und mathabx vermeiden:
\def\delsym#1{%
%  \expandafter\let\expandafter\origsym\expandafter=\csname#1\endcsname
%  \expandafter\let\csname orig#1\endcsname=\origsym
  \expandafter\let\csname#1\endcsname=\relax
}

%\usepackage{pifont}
%\usepackage{bbding}
%\delsym{Asterisk}
%\delsym{Sun}\delsym{Mercury}\delsym{Venus}\delsym{Earth}\delsym{Mars}
%\delsym{Jupiter}\delsym{Saturn}\delsym{Uranus}\delsym{Neptune}
%\delsym{Pluto}\delsym{Aries}\delsym{Taurus}\delsym{Gemini}
%\delsym{Rightarrow}
%\usepackage{mathabx} - Ueberschreibt leider zu viel - und die \le-Zeichen usw. sehen nicht gut aus!


%Fallback-Schriftart
%\usepackage{lmodern}  % Latin Modern Fonts sind die Nachfolger von Computer Modern, den LaTeX-Standardfonts
%Quelle: http://homepage.ruhr-uni-bochum.de/Georg.Verweyen/pakete.html
%Allerdings sieht diese Schritart in Diplomarbeiten fuer Fliesstext auch nicht besonders schoen aus.
%Trotzdem ist sie fuer Programmcode gut geeignet

%Schriftart fuer die Ueberschriften - ueberschreibt lmodern
\usepackage[scaled=.95]{helvet} %fuer englische Texte .95 auf .90 aendern

%Schriftart fuer Programmcode - ueberschreibt lmodern
%Falls auskommentiert, wird die Standardschriftart genommen
%\usepackage[scaled=.92]{luximono} % Fuer schreibmaschinenartige Schluesselwoerter in den Listings - geht bei alten Installationen nicht, da einige Fontshapes (<>=) fehlen
%\usepackage{courier} 

% Tolle Schriften...
%\usepackage{helvet}
%\usepackage{palatino}
\usepackage[osf]{libertine}
% Für Schreibschrift würde tun, muss aber ned
%\usepackage{mathrsfs} %  \mathscr{ABC}


%Schriftart fuer den Fliesstext - ueberschreibt lmodern
%
\usepackage[osf]{mathpazo} %ftp://ftp.dante.de/tex-archive/fonts/mathpazo/ - Tipp aus DE-TEX-FAQ 8.2.1
%Bringt Palantino, osf = Minuskel-Ziffern
%
%\usepackage{charter} %Charter fuer englsiche Texte
%
%\usepackage{mathptmx} %Times fuer englische Texte. Sieht nicht sooo gut aus.

\usepackage[T1]{fontenc}

\usepackage{microtype} % optischer Randausgleich - bei miktex gleich dabei - bei linux von
%  http://www.ctan.org/tex-archive/macros/latex/contrib/microtype/
%  herunterladen 
%Falls bei einer Silbentrennung ploetzlich eine ganze Zeile fehlt (passiert unter Windows XP mit MikTex 2.5 und foxit reader als pdfreader
%\usepackage{pdfcprot}
%ausprobieren. Dieses erzeugt allerdings nur für Palatino (in dieser Vorlage die Default-Schrift) einen guten optischen Randausgleich
%Falls alle Stricke reissen, muss leider auf den optischen Randausgleich verzichtet werden.

%fuer microtype
%tracking=true muss als Parameter des microtype-packages mitgegeben werden
%
%Deaktiviert, da dies bei Algorithmen seltsam aussieht
%
%\DeclareMicrotypeSet*[tracking]{my}{ font = */*/*/sc/* }% 
%\SetTracking{ encoding = *, shape = sc }{ 45 }% Hier wird festgelegt,
            % dass alle Passagen in Kapitälchen automatisch leicht
            % gesperrt werden.
			% Quelle: http://homepage.ruhr-uni-bochum.de/Georg.Verweyen/pakete.html
