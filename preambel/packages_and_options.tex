% $Id: packages_and_options.tex 81 2009-04-09 11:18:17Z koppor $

%in dieser Datei werden zuerst die benoetigten Pakete eingebunden und
%danach diverse Optionen gesetzt

\usepackage[ngerman]{babel} %"Literatur" statt "Literature", Nachfolger von ngerman.sty
%f�r englische Texte:
%\usepackage[american]{babel}
%Hinweise zu weiteren, notwendigen Umstellungen in README.txt beachten
%
\usepackage[ansinew]{inputenc} %ansinew fuer Windows, latin1 fuer Linux, da "utf8" noch nicht von allen Windows-Editoren unterstützt 
%
 % erweitertes Enumerate
\usepackage{paralist}
%
\usepackage[automark]{scrpage2} % fancyheadings fuer koma
%
%Versionskontrolle
%-----------------
%SVN. F�r die final version "draft" durch "final" ersetzen
%\usepackage[draft,eso-foot]{svninfo}
%
%Falls rcs oder cvs als Versionskontrolle verwendet wird, dann anstattdessen
%\usepackage{rcs}
%verwenden
%
%
%Mathematik
%----------
\usepackage[fleqn,leqno]{amsmath} % Viele Mathematik-Sachen: Doku: /usr/share/doc/texmf/latex/amsmath/amsldoc.dvi.gz
%fleqn (=Gleichungen linksb�ndig platzieren) funktioniert nicht direkt. Es muss noch ein Patch gemacht werden:
\addtolength\mathindent{1em}%work-around ams-math problem with align and 9 -> 10
\usepackage{mathtools} %fixes bugs in AMS math
%
%
\usepackage{xspace}

%\usepackage{appendix}
%[toc,page,title,header]

\usepackage{graphicx}%Parameter "pdftex" unnoetig
\graphicspath{{graphics/}}

\usepackage{array} %increases tex's buffer size and enables ``>'' in tablespecs
\usepackage{longtable}

\usepackage{url}

%\usepackage{makeidx}
%makeidx ist out -> http://xindy.sf.net verwenden

\usepackage{ifthen} %fuer if-then-else @ commands.tex

%\usepackage[Bjarne]{fncychap}
%\usepackage[Lenny]{fncychap}

\usepackage{diplomtitel/diplomtitel}

\usepackage{listings}
%Alternative zu Listings ist fancyvrb. Kann auch beides gleichzeitig benutzt werden.
\usepackage{fancyvrb}
%\fvset{fontsize=\small} %Groesse fuer den Fliesstext. Falls deaktiviert: \normalsize
\DefineShortVerb{\�} %Somit kann im Text ganz einfach �verbatim� text gesetzt werden.
\RecustomVerbatimEnvironment{Verbatim}{Verbatim}{fontsize=\footnotesize}
\RecustomVerbatimCommand{\VerbatimInput}{VerbatimInput}{fontsize=\footnotesize}

%Bildunterschriften bei floats genauso formatieren wie bei Listings
%Anpassung wird unten bei den newfloat-Deklarationen vorgenommen
\usepackage{caption}

%Ermoeglicht es, Abbildungen um 90 Grad zu drehen
\usepackage{lscape}
%Alternatives Paket: rotating. Allerdings wird hier nur das Bild gedreht, w�hrend bei lscape auch die PDF-Seite gedreht wird.

%Fuer listings
% Wird f�r fancyvrb und f�r lstlistings verwendet
%zust�zlich f�r den Paramter [H] = Floats WIRKLICH da wo sie deklariert wurden paltzieren - ganz ohne Kompromisse
%floatrow ist der Nachfolger von float
\usepackage{floatrow} 

%Fuer Abbildungen innerhalb von Abbildungen
%Ersetzt das Paket subfigure
%\usepackage{subfig}

%Fuer Tabellen mit Variablen Spaltenbreiten
%\usepackage{tabularx}
%\usepackage{tabulary}

\usepackage[ngerman]{varioref}
%Falls die Seitenzahl bei einer Referenz auf eine Abbildung nur dann angegeben werden soll,
%  falls sich die Abbildung nicht auf der selben Seite befindet...
%
\usepackage{booktabs}
%Noch schoenere Tabellen als mit booktabs bekommt man mit dem Tipp von
% http://www.zvisionwelt.de/downloads.html
% hin
%
%\usepackage[section]{placeins}
%
%
%Fuer Graphiken. Allerdings funktioniert es nicht zusammen mit pdflatex
%\usepackage{gastex} % \tolarance kann dann nicht mehr umdefiniert werden
%
%\usepackage{multicol}
%\usepackage{setspace} % kollidiert mit diplomarbeit.sty
\usepackage[hyperref,dvipsnames]{xcolor} %Parameter "pdftex" ist unnoetig
%
%http://www.tex.ac.uk/cgi-bin/texfaq2html?label=floats
%\usepackage{flafter} %floats IMMER nach ihrer Deklaration platzieren
%

%schoene TODOs
\usepackage{todonotes}

% neue Pakete bitte vor hyperref einbinden. Insbesondere bei Verwendung des
% Pakets "index" wichtig, da sonst die Referenzierung nicht funktioniert.
% Für die Indizierung selbst ist unter http://xindy.sourceforge.net
% ein gutes Tool zu erhalten

%ggf.in der Endversion komplett rausnehmen. dann auch \href in commands.tex aktivieren
\usepackage[
%pdftitle, pdfauthor, pdfkeywords und pdfsubject werden in ausarbeitung.tex gesetzt
%            pdfstartpage=7,
            pdfpagelabels=true, % so werden die Seitennummern auch ins PDF uebernommen
            a4paper,
            %backref=section,  %bibliographic back-references
            bookmarks=true,
            bookmarksnumbered=true,
            bookmarksopen=true,
            bookmarksopenlevel=1,
            breaklinks=true, %default: false
            %frenchlinks=true, %koennte gut aussehen, aber xpdf zeigt keine Aenderung
            colorlinks=true,
            hyperindex=true,
            linkcolor=black, %blue, %default: red
            %pagecolor=black, %blue, %default: red - removed in recent hyperref versions
            citecolor=black, %Mulberry,%DarkOrchid, %default: green
            urlcolor=black, %blue, default: cyan
            pdfpagelayout=SinglePage,
%            pdfstartpage=13
			%Treiber muss nicht unbedingt angegeben werden.
            %ps2pdf
            %dvips
            %dvipdf
			%pdftex
            ]{hyperref} %siehe auch: "Praktisches LaTeX" - www.itp.uni-hannover.de/~kreutzm
%
%Zur Darstellung von Algorithmen
\usepackage[chapter]{algorithm} % algorithm muss nach hyperref geladen werden
\usepackage{algpseudocode}
%

%Schriften
%%%
%
\automark[section]{chapter}
\setkomafont{pageheadfoot}{\normalfont\sffamily}
\setkomafont{pagenumber}{\normalfont\rmfamily}
%\setheadsepline[.4pt]{.4pt} %funktioniert nicht: Alle Linien sind hier weg
%
%%%

%%%
%
\ifenglisch
% Fuer englische Texte sind serifenhafte Ueberschriften gut. Deshalb hier der Befehl zum Aktivieren von serifenhaften Ueberschriften
\setkomafont{disposition}{\normalfont\rmfamily}

% Bei englisschen Texten das Label (optionaler Eintrag bei \item) bei description-Umgegungen nur auf fett und nicht fett+serifenlos stellen.
\setkomafont{descriptionlabel}{\normalfont\bfseries}
\fi
%
%%%

%%%
% Fuer deutsche Texte: Weniger Silbentrennung, mehr Abstand zwischen den Woertern
\ifdeutsch
\setlength{\emergencystretch}{3em} % Silbentrennung reduzieren durch mehr frei Raum zwischen den Worten
\fi
%%%

%Symbole
%--------
%\usepackage[geometry]{ifsym} % \BigSquare
%\usepackage{mathabx}
%\usepackage{stmaryrd} %fuer \ovee, \owedge, \otimes
%\usepackage{marvosym} %fuer \Writinghand %patched to not redefine \Rightarrow
%\usepackage{mathrsfs} %mittels \mathscr{} schoenen geschwungenen Buchstaben erzeugen
%\usepackage{calrsfs} %\mathcal{} ein bisserl dickeren buchstaben erzeugen - sieht net so gut aus.
                      %durch mathpazo ist das schon definiert
\usepackage{amssymb}

%name-clashes von marvosym und mathabx vermeiden:
\def\delsym#1{%
%  \expandafter\let\expandafter\origsym\expandafter=\csname#1\endcsname
%  \expandafter\let\csname orig#1\endcsname=\origsym
  \expandafter\let\csname#1\endcsname=\relax
}

%\usepackage{pifont}
%\usepackage{bbding}
%\delsym{Asterisk}
%\delsym{Sun}\delsym{Mercury}\delsym{Venus}\delsym{Earth}\delsym{Mars}
%\delsym{Jupiter}\delsym{Saturn}\delsym{Uranus}\delsym{Neptune}
%\delsym{Pluto}\delsym{Aries}\delsym{Taurus}\delsym{Gemini}
%\delsym{Rightarrow}
%\usepackage{mathabx} - Ueberschreibt leider zu viel - und die \le-Zeichen usw. sehen nicht gut aus!


%Fallback-Schriftart
%\usepackage{lmodern}  % Latin Modern Fonts sind die Nachfolger von Computer Modern, den LaTeX-Standardfonts
%Quelle: http://homepage.ruhr-uni-bochum.de/Georg.Verweyen/pakete.html
%Allerdings sieht diese Schritart in Diplomarbeiten fuer Fliesstext auch nicht besonders schoen aus.
%Trotzdem ist sie fuer Programmcode gut geeignet

%Schriftart fuer die Ueberschriften - ueberschreibt lmodern
\ifdeutsch
\usepackage[scaled=.95]{helvet}
\else
\usepackage[scaled=.90]{helvet}
\fi

%Schriftart fuer Programmcode - ueberschreibt lmodern
%Falls auskommentiert, wird die Standardschriftart genommen
%\usepackage[scaled=.92]{luximono} % Fuer schreibmaschinenartige Schluesselwoerter in den Listings - geht bei alten Installationen nicht, da einige Fontshapes (<>=) fehlen
%\usepackage{courier} 

% Tolle Schriften...
%\usepackage{helvet}
%\usepackage{palatino}
%\usepackage[osf]{libertine}
% Für Schreibschrift würde tun, muss aber ned
%\usepackage{mathrsfs} %  \mathscr{ABC}


%Schriftart fuer den Fliesstext - ueberschreibt lmodern
%
\ifdeutsch
\usepackage[osf]{mathpazo} %ftp://ftp.dante.de/tex-archive/fonts/mathpazo/ - Tipp aus DE-TEX-FAQ 8.2.1
\fi
%Bringt Palantino, osf = Minuskel-Ziffern
%
%\usepackage{charter} %Charter fuer englsiche Texte
%\linespread{1.05} % Durchschuss für Charter leicht erhöhen
%
%\usepackage{mathptmx} %Times fuer englische Texte. Sieht nicht sooo gut aus.
\ifenglisch
\usepackage{lmodern}
\fi

\usepackage[T1]{fontenc}


% optischer Randausgleich - bei miktex gleich dabei - bei linux von
%  http://www.ctan.org/tex-archive/macros/latex/contrib/microtype/
%  herunterladen 
\usepackage{microtype}
%Falls bei einer Silbentrennung ploetzlich eine ganze Zeile fehlt (passiert unter Windows XP mit MikTex 2.5 und foxit reader als pdfreader
%\usepackage{pdfcprot}
%ausprobieren. Dieses erzeugt allerdings nur für Palatino (in dieser Vorlage die Default-Schrift) einen guten optischen Randausgleich
%Falls alle Stricke reissen, muss leider auf den optischen Randausgleich verzichtet werden.

%fuer microtype
%tracking=true muss als Parameter des microtype-packages mitgegeben werden
%
%Deaktiviert, da dies bei Algorithmen seltsam aussieht
%
%\DeclareMicrotypeSet*[tracking]{my}{ font = */*/*/sc/* }% 
%\SetTracking{ encoding = *, shape = sc }{ 45 }% Hier wird festgelegt,
            % dass alle Passagen in Kapitälchen automatisch leicht
            % gesperrt werden.
			% Quelle: http://homepage.ruhr-uni-bochum.de/Georg.Verweyen/pakete.html



%Bugfix packages
\usepackage{fixltx2e} %Bereinigt einige Ungereimtheiten, die auf Grund von Rueckwaertskompatibilitaet beibahlten wurden.
%\usepackage{mparhack} %Fixt die Position von marginpars (die in DAs selten bis gar nicht gebraucht werden}
%\usepackage{ellipsis} %Fixt die Abstaende vor \ldots. Wird wohl auch nicht benoetigt.

%%%%%%%%%%%%%%%%%%%%%%%%%%%%%%%%%%%%%%%%%%%%%%%%%%%%%%%%%%%%%%%%%%%%%%%%%%%%%%
% Rand                                                                       %
%%%%%%%%%%%%%%%%%%%%%%%%%%%%%%%%%%%%%%%%%%%%%%%%%%%%%%%%%%%%%%%%%%%%%%%%%%%%%%
%Viele Moeglichkeiten, die Raender im Dokument einzustellen.
%Satzspiegel neu berechnen. Dokumentation dazu ist in "scrguide.pdf" von KOMA-Skript zu finden
%  Optionen werden bei \documentclass[] in ausarbeitung.tex mitgegeben.
\typearea[current]{current} %neu berechnen, da neue Schrift eingebunden

%\usepackage{a4}
%\usepackage{a4wide}
%\areaset{170mm}{277mm} %a4:29,7hochx21mbreit

%Wer die Masse direkt eingeben moechte:
%Bei diesem Beispiel wird die Regel nicht beachtet, dass der innere Rand halb so gross wie der aussere Rand und der obere Rand halb so gross wie der untere Rand sein sollte
%\usepackage[inner=2.5cm, outer=2.5cm, includefoot, top=3cm, bottom=1.5cm]{geometry}




%%%%%%%%%%%%%%%%%%%%%%%%%%%%%%%%%%%%%%%%%%%%%%%%%%%%%%%%%%%%%%%%%%%%%%%%%%%%%%
% Optionen                                                                   %
%%%%%%%%%%%%%%%%%%%%%%%%%%%%%%%%%%%%%%%%%%%%%%%%%%%%%%%%%%%%%%%%%%%%%%%%%%%%%%


%Skip=0 funktioniert nicht
\captionsetup{format=hang,labelfont=bf,justification=justified,singlelinecheck=false,skip=0pt}

%neue float Umgebung fuer Listings, die mittels fancyvrb gesetzt werden sollen
\floatstyle{ruled}
\newfloat{Listing}{tbp}{code}[chapter]
\newfloat{Algorithmus}{tbp}{alg}[chapter]

%amsmath
%\numberwithin{equation}{section}
%\renewcommand{\theequation}{\thesection.\Roman{equation}}

%pdftex
\pdfcompresslevel=9

%Tabellen (array.sty)
\setlength{\extrarowheight}{1pt}


% Andere Kapitelueberschriften
% falls einem der Standard von KOMA nicht gefaellt...
% Falls man zurück zu KOMA moechte, dann muss jede der vier folgenden Moeglichkeiten deaktiviert sein.

% 1. Moeglichkeit
%\usepackage[Sonny]{fncychap}

% 2. Moeglichkeit
\iffalse
\usepackage[Bjarne]{fncychap}
\ChNameVar{\Large\sf} \ChNumVar{\Huge} \ChTitleVar{\Large\sf}
\ChRuleWidth{0.5pt} \ChNameUpperCase
\fi

%Variante der 2. Moeglichkeit
\iffalse
\usepackage[Rejne]{fncychap}
\ChNameVar{\centering\Huge\rm\bfseries}
\ChNumVar{\Huge}
 \ChTitleVar{\centering\Huge\rm}
\ChNameUpperCase
\ChTitleUpperCase
\ChRuleWidth{1pt}
\fi

% 3. Moeglichkeit
\iffalse
\usepackage{fncychap}
\ChNameUpperCase
\ChTitleUpperCase
\ChNameVar{\raggedright\normalsize} %\rm
\ChNumVar{\bfseries\Large}
\ChTitleVar{\raggedright\Huge}
\ChRuleWidth{1pt}
\fi

% 4. Moeglichkeit
% Zur Aktivierierung "\iffalse" und "\fi" auskommentieren
% Innen drin kann man dann noch zwischen
%   * serifenloser Schriftart (eingestellt)
%   * serifenhafter Schriftart (wenn kein zusaetzliches Kommando aktiviert ist) und
%   * Kapitälchen wählen
\iffalse
\makeatletter
%\def\thickhrulefill{\leavevmode \leaders \hrule height 1ex \hfill \kern \z@}

%Fuer Kapitel mit Kapitelnummer
\def\@makechapterhead#1{%
  \vspace*{10\p@}%
  {\parindent \z@ \raggedright \reset@font
			%Default-Schrift: Serifenhaft (gut fuer englische Dokumente)
            %A) Fuer serifenlose Schrift:
            \fontfamily{phv}\selectfont
			%B) Fuer Kapitaelchen:
			%\fontseries{m}\fontshape{sc}\selectfont
            %C) Fuer ganz "normale" Schrift:
            %\normalfont 
			%
			\Large \@chapapp{} \thechapter
        \par\nobreak\vspace*{10\p@}%
        \interlinepenalty\@M
    {\Huge\bfseries\baselineskip3ex
	%Fuer Kapitaelchen folgende Zeile aktivieren:
	%\fontseries{m}\fontshape{sc}\selectfont
	#1\par\nobreak}
    \vspace*{10\p@}%
\makebox[\textwidth]{\hrulefill}%    \hrulefill alone does not work
    \par\nobreak
    \vskip 40\p@
  }}

  %Fuer Kapitel ohne Kapitelnummer (z.B. Inhaltsverzeichnis)
  \def\@makeschapterhead#1{%
  \vspace*{10\p@}%
  {\parindent \z@ \raggedright \reset@font
            \normalfont \vphantom{\@chapapp{} \thechapter}
        \par\nobreak\vspace*{10\p@}%
        \interlinepenalty\@M
    {\Huge \bfseries %
	%Default-Schrift: Serifenhaft (gut fuer englische Dokumente)
    %A) Fuer serifenlose Schrift folgende Zeile aktivieren:
    \fontfamily{phv}\selectfont
	%B) Fuer Kapitaelchen folgende Zeile aktivieren:
	%\fontseries{m}\fontshape{sc}\selectfont
	#1\par\nobreak}
    \vspace*{10\p@}%
\makebox[\textwidth]{\hrulefill}%    \hrulefill does not work
    \par\nobreak
    \vskip 40\p@
  }}
%
\makeatother
\fi


%Minitoc-Einstellungen
%\dominitoc
%\renewcommand{\mtctitle}{Inhaltsverzeichnis dieses Kapitels}

% Disable single lines at the start of a paragraph (Schusterjungen)
\clubpenalty = 10000
%
% Disable single lines at the end of a paragraph (Hurenkinder)
\widowpenalty = 10000 \displaywidowpenalty = 10000

%http://groups.google.de/group/de.comp.text.tex/browse_thread/thread/f97da71d90442816/f5da290593fd647e?lnk=st&q=tolerance+emergencystretch&rnum=5&hl=de#f5da290593fd647e
%Mehr Infos unter http://www.tex.ac.uk/cgi-bin/texfaq2html?label=overfull
\tolerance=2000
\setlength{\emergencystretch}{3pt}   % kann man evtl. auf 20 erhoehen
\setlength{\hfuzz}{1pt}

%fuer listings.sty
\lstset{language=XML,
        showstringspaces=false,
        extendedchars=true,
        basicstyle=\footnotesize\ttfamily,
        commentstyle=\slshape,
        stringstyle=\ttfamily, %Original: \rmfamily, damit werden die Strings im Quellcode hervorgehoben		zusaetzlich evtl.: \scshape oder \rmfamily durch \ttfamily ersetzen. Dann sieht's aus, wie bei fancyvrb
        breaklines=true,
        breakatwhitespace=true,
        columns=flexible,
        aboveskip=0mm, %deaktivieren, falls man lstlistings direkt als floating object benutzt (\begin{lstlisting}[float,...])
        belowskip=0mm, %deaktivieren, falls man lstlistings direkt als floating object benutzt (\begin{lstlisting}[float,...])
        captionpos=b
}
\renewcommand{\lstlistlistingname}{Verzeichnis der Listings}

%fuer algorithm.sty: - falls Deutsch und nicht Englisch. Falls Englisch als Sprache gew�hlt wurde, bitte die folgenden beiden Zeilen auskommentieren.
\floatname{algorithm}{Algorithmus}
\renewcommand{\listalgorithmname}{Verzeichnis der Algorithmen}

%Fuer Palatino (mathpazo.sty): richtiges Euro-Zeichen
%  wird von marvosym definiert - ist nicht mehr eingebunden
%Alternative: \usepackage{eurosym}
\newcommand{\EUR}{\ppleuro}

\automark[section]{chapter}
\setkomafont{pagehead}{\normalfont\sffamily} %gilt auch f"ur Fu"s
\setkomafont{pagenumber}{\normalfont\rmfamily}
%\setheadsepline[.4pt]{.4pt} %funktioniert nicht: Alle Linien sind hier weg

%Fuer englische Texte sind serifenhafte Ueberschriften gut. Deshalb hier der Befehl zum Aktivieren von serifenhaften Ueberschriften
%\setkomafont{disposition}{\normalfont\rmfamily}

%Fuer deutsche Texte: Weniger Silbentrennung, mehr Abstand zwischen den Woertern
\setlength{\emergencystretch}{3em} % Silbentrennung reduzieren durch mehr frei Raum zwischen den Worten

%Float-placements - http://dcwww.camd.dtu.dk/~schiotz/comp/LatexTips/LatexTips.html#figplacement
%and http://people.cs.uu.nl/piet/floats/node1.html
\renewcommand{\topfraction}{0.85}
\renewcommand{\bottomfraction}{0.95}
\renewcommand{\textfraction}{0.1}
\renewcommand{\floatpagefraction}{0.75}
%\setcounter{totalnumber}{5}

%Bei Gleichungen nur dann die Nummer zeigen, wenn die Gleichung auch referenziert wird
\mathtoolsset{showonlyrefs}

%Rueckverweise aus dem Literaturverzeichnis
\usepackage[hyperpageref]{backref}
%Deutscher Text
\newcommand{\babelbackrefnotcited}{\relax}
\newcommand{\babelbackrefcitedsingle}[1]{(Zitiert auf Seite~#1)}
\newcommand{\babelbackrefcitedmulti}[1]{(Zitiert auf den Seiten~#1)}
\newcommand{\babelbackrefand}{und}
\iffalse
%Englischer Text
\newcommand{\babelbackrefnotcited}{\relax}
\newcommand{\babelbackrefcitedsingle}[1]{(Cited on page~#1)}
\newcommand{\babelbackrefcitedmulti}[1]{(Cited on pages~#1)}
\newcommand{\babelbackrefand}{und}
\fi
%Tweak backref
\renewcommand{\backreftwosep}{ \babelbackrefand~} % seperate 2 pages
\renewcommand{\backreflastsep}{ \babelbackrefand~} % seperate last of longer list
\renewcommand*{\backref}[1]{} % Standard deaktivieren
\renewcommand*{\backrefalt}[4]{% 
\ifcase #1 %
\babelbackrefnotcited%
\or
\babelbackrefcitedsingle{#2}%
\else
\babelbackrefcitedmulti{#2}%
\fi}

\setcounter{tocdepth}{2}

% Seitengroessen - Gegen Schusterjungen und Hurenkinder...
\newcommand{\largepage}{\enlargethispage{\baselineskip}}
\newcommand{\shortpage}{\enlargethispage{-\baselineskip}}
